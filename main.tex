\documentclass{report}
\usepackage{graphicx}
\graphicspath{ {images/} }
\usepackage{tabularx}
\usepackage{mathtools ,amssymb ,amsthm} % imports  amsmath
\usepackage{nameref}
\usepackage{hyperref}
\usepackage{amsmath}
\usepackage{tgbonum}
\usepackage[table]{xcolor}
% use package for pretty circle
\usepackage{tikz}
\newcommand*\circled[1]{\tikz[baseline=(char.base)]{
    \node[shape=circle,draw,inner sep=2pt] (char) {#1};}}

% use package for line in matrix
\usepackage{amsmath}
    \makeatletter
    \renewcommand*\env@matrix[1][*\c@MaxMatrixCols c]{%
      \hskip -\arraycolsep
      \let\@ifnextchar\new@ifnextchar
      \array{#1}}
    \makeatother


\usepackage{subfiles} % Best loaded last in the preamble

\title{LINAL - Spick}
\author{Johanna Koch}
\date{ }

\begin{document}
{\fontfamily{lmss}\selectfont

    \maketitle
    \tableofcontents
    
    \subfile{chapters/sw01_vektoren_matrizen_gleichungssysteme.tex}
    \subfile{chapters/sw02_lin_gleich_1.tex}

}
\end{document}

% \begin{pmatrix}0\\0\end{pmatrix}
% \noindent\rule{8cm}{0.4pt} \\

\iffalse
\begin{tabularx}{0.8\textwidth} { 
    >{\centering\arraybackslash}X 
    >{\centering\arraybackslash}X  }
    \begin{math}
        {}
    \end{math}
    &
    \begin{math}
        {}
    \end{math}
    \\ [7pt]
    \begin{math}
        {}
    \end{math}
    &
    \begin{math}
        {}
    \end{math}
    \\ [7pt]
\end{tabularx}
\fi