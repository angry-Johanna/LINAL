\documentclass[../main.tex]{subfiles}


\begin{document}
%%%%%%%%%%%%%%%%%%%%%%%%%%%%%%%%%%%%%%%%%%%%%%%%%%%%%%%%%%%%%%%%%%%%%%%%%%%%%%%%%%%%%%%%%%%%%%%%%%%%%%%%%%%%%%%%%%%
%                                                                                                                 %
% Lineare Gleichungssysteme lösen 2: Gausselimination und Optimierung beim Ermittelnder Lösung durch LU-Zerlegung %
%                                                                                                                 %
%%%%%%%%%%%%%%%%%%%%%%%%%%%%%%%%%%%%%%%%%%%%%%%%%%%%%%%%%%%%%%%%%%%%%%%%%%%%%%%%%%%%%%%%%%%%%%%%%%%%%%%%%%%%%%%%%%%

\chapter{Lineare Gleichungssysteme lösen 2: Gausselimination und Optimierung beim Ermittelnder Lösung durch LU-Zerlegung}

\section{Einheitsmatrix}
Eine quadratische Matrix, in deren Hauptdiagonale nur Einsen stehen und deren sämtliche
andere Einträge verschwinden (Null sind), nennt mann Einheitsmatrix (engl. identity matrix).
Hat so eine Matrix die Dimension $n\times n$, so schreibt man für sie eins der Symbole $_En$ oder $I_n$. \\ [7pt]

$\mathbf{E_4}=\mathbf{I_4}=
\begin{bmatrix}
    1 & 0 & 0 & 0 \\
    0 & 1 & 0 & 0 \\
    0 & 0 & 1 & 0 \\
    0 & 0 & 0 & 1 \\
\end{bmatrix}$ \\ [7pt]
Eine Einheitsmatrix $\mathbf{E}$ hat die Eigenschaft, dass $\mathbf{EA}=\mathbf{AE}=\mathbf{A}$.
Einheitsmatrizen sind also neutrale Elemente bezgl. der Matrizenmultiplikation.

\section{Transponierte Matrix}
Eine Matrix $\mathbf{A}$ ist transponiert, wenn man die Matrix an ihrer Hauptdiagonalen spiegelt.
In anderen Worten: Eine Matrix $\mathbf{A}$ ist transponiert, wenn man ihre Zeilen und Spalten von der 
Matrix $\mathbf{A}$ vertauscht. Transponierte Matrix $\mathbf{A}$ schreibt man $\mathbf{A^T}$. 
Es gibt einen wichtigen Zusammenhang zwischen Transposition und Matrizenmultiplikation: \\ [7pt]

$(\mathbf{A} \cdot \mathbf{B})^T=\mathbf{B}^T \cdot \mathbf{A}^T$ \\ [7pt]

$\mathbf{A} = 
\begin{bmatrix}
    4 & 5 & 0 \\
    -2 & -2 & 8 \\
    -1 & 4 & 3 \\
    7 & 0 & -6 \\
\end{bmatrix},
\mathbf{A}^T = 
\begin{bmatrix}
    4 & -2 & -1 & 7 \\
    5 & -2 & -4 & 0 \\
    0 & 8 & 3 & -6 \\
\end{bmatrix}$

\section{Invertierbare oder reguläre Matrix}
Sei $\mathbf{A}$ eine quadratische $n\times n$-Matrix. $\mathbf{A}$ heisst eine invertierbare oder reguläre 
(auch non-singuläre) Matrix, wenn es eine $n\times n$-Matrix $\mathbf{B}$ existiert so, dass: \\ [7pt]
$\mathbf{A} \cdot \mathbf{B} = I_n = E_n = \mathbf{B} \cdot \mathbf{A}$ \\ [7pt]
Dann nennt man $\mathbf{B}$ die inverse Matrix zu $\mathbf{A}$ und schreibt sie als $\mathbf{A}^{-1}$

\subsection{Beispiel}
$A=
\begin{bmatrix}
    1 & 3 \\
    2 & 7 \\
\end{bmatrix},
B=A^{-1}=
\begin{bmatrix}
    7 & -3 \\
    -2 & 1 \\
\end{bmatrix}$ \\ [7pt]

$\begin{bmatrix}
    1 & 3 \\
    2 & 7 \\
\end{bmatrix} \cdot
\begin{bmatrix}
    7 & -3 \\
    -2 & 1 \\
\end{bmatrix}
=
\begin{bmatrix}
    1 & 0 \\
    0 & 1 \\
\end{bmatrix}$

\section{Rechenregeln}
$\mathbf{AB}\cdot(\mathbf{AB})^{-1}=I$ \\ [7pt]
$(\mathbf{AB})^{-1} = \mathbf{A^{-1} \cdot B^{-1}}$ \\ [7pt]
$(\mathbf{A^{-1} \cdot B^{-1}})\cdot \mathbf{AB} = I$

\section{Permutationsmatrix}
Die Permutationsmatrix $\mathbf{P}$ ist eine quadratische $(n\times n)$-Matrix, die in jeder Zeile und jeder
Spalte genau einen Eintrag von eins (1) und an anderer Stelle Nullen (0) hat. Die
Multiplikation jeder Matrix $\mathbf{A}$ mit der Matrix $\mathbf{P}$ führt zum Vertauschen der Zeilen oder Spalten
der Matrix $\mathbf{A}$. Zur Permutation der Zeilen oder Spalten der Matrix $\mathbf{A}$ wird $\mathbf{PA}$
(Vormultiplizieren) oder $\mathbf{AP}$ (Nachmultiplizieren) durchgeführt. \\
Mit anderen Worten: Die Permutationsmatrix $\mathbf{P}$ ist eine Einheitsmatrix, wo die Zeilen umgeordnet sind.

\subsection{Beispiel}
$I=\begin{bmatrix}
    1 & 0 \\
    0 & 1 \\
\end{bmatrix} \rightarrow
\begin{bmatrix}
    0 & 1 \\
    1 & 0 \\
\end{bmatrix} = $ Permutationsmatrix \\ [7pt]

$\begin{bmatrix}
    2 & 3 \\
    4 & -5 \\
\end{bmatrix} \cdot
\begin{bmatrix}
    0 & 1 \\
    1 & 0 \\
\end{bmatrix} = 
\begin{bmatrix}
    4 & -5 \\
    2 & 3 \\
\end{bmatrix}$

\section{TODO: LU-Zerlegung}



\end{document}