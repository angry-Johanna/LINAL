\documentclass[../main.tex]{subfiles}


\begin{document}
%%%%%%%%%%%%%%%%%%%%%%%%%%%%%%%%%%%%%
%                                   %
% Lineare Gleichungssysteme lösen 3 %
%                                   %
%%%%%%%%%%%%%%%%%%%%%%%%%%%%%%%%%%%%%

\chapter{Lineare Gleichungssysteme lösen 3: Optimierung der Lösungsbestimmung durch Gauss-Jordan-Elimination, Inverse Matrix, Determinante}

\section{Gauss-Jordan-Algorithmus Vorgehensweise}
\textbf{Startpunkt}: Nachdem man ein LGS auf Zeilenstufenform gebracht hat, kann man es optional
noch weiter mit Hilfe von elementaren Zeilenumformungen modifizieren. Daraus lässt sich dann
die Lösung des LGS direkt (ohne Rückwärtseinsetzen) ablesen. \\

\textbf{Wichtig!!} Das LGS bzw. dessen erweiterte Koeffizientenmatrix (EKM) ist bereits in
Zeilenstufenform. Unser Ziel ist es, in den Spalten oberhalb der Leitkoeffizienten Nullen zu
erzeugen:

\begin{itemize}
    \item Während wir beim Erzeugen der Zeilenstufenform (Gauss-Verfahren) das LGS bzw. dessen
    EKM von oben nach unten und von links nach rechts durchgearbeitet haben, gehen wir
    nun von rechts nach links und von unten nach oben vor.
    \item Es werden keine Zeilen mehr umgetauscht. Der Leitkoeffizient, der gerade daran ist, ist
    zwangsläufig das Pivotelement.
    \item Alle diese Pivotelemente werden als Erstes (durch Multiplikation der Zeile mit dem
    Kehrwert) auf den Wert eins gebracht.
\end{itemize}

TODO: Beispiel??

\section{Gauss-Jordan-Algorithmus inverse Matrix}
\subsection{Beispiel}
Berechne die inverse Matrix von $\mathbf{A}$ mit dem Gauss-Jordan Algorithmus. \\

$\mathbf{A}=
\begin{bmatrix}
    1 & 3 \\
    2 & 7 \\
\end{bmatrix}, 
\begin{bmatrix}
    1 & 3 \\
    2 & 7 \\
\end{bmatrix} \cdot
\begin{bmatrix}
    a & b \\
    c & d \\
\end{bmatrix} =
\begin{bmatrix}
    1 & 0 \\
    0 & 1 \\
\end{bmatrix}$ \\ [7pt]

$\begin{bmatrix}[cc|cc]
    1 & 3 & 1 & 0 \\
    2 & 7 & 0 & 1 \\
\end{bmatrix} \rightarrow II - 2I \rightarrow
\begin{bmatrix}[cc|cc]
    1 & 3 & 1 & 0 \\
    0 & 1 & -2 & 1 \\
\end{bmatrix}$ \\ [7pt]

$ \rightarrow I - 3II \rightarrow
\begin{bmatrix}[cc|cc]
    1 & 0 & 7 & -3 \\
    0 & 1 & -2 & 1 \\
\end{bmatrix}$ \\ [7pt]
Probe: 
$\begin{bmatrix}
    1 & 3 \\
    2 & 7 \\
\end{bmatrix} \cdot
\begin{bmatrix}
    7 & -3 \\
    -2 & 1 \\
\end{bmatrix} =
\begin{bmatrix}
    1 & 0 \\
    0 & 1 \\
\end{bmatrix}$

\section{Determinante}
Eine reguläre Matrix ist invertiebar und eine nicht reguläre Matrix ist nicht invertiebar. 

\textbf{Wichtig!!} Der algebraische Test für Invertierbarkeit der Matrix $\mathbf{A}$ ist die Determinante von $\mathbf{A}$:
$det(\mathbf{A})$ darf nicht Null sein.

Die Determinante ist der Flächeninhalt bzw Volumeninhalt der 2x2Matrix bzw 3x3Matrix im geometrischen Raum.

\subsection{Determinante einer 2x2-Matrix}
Gegeben:
$\mathbf{A} =
\begin{bmatrix}
    a & b \\
    c & d \\
\end{bmatrix}$ \\
$det(\mathbf{A}) = |\mathbf{A}| = ad - bc$

\subsection{Fläche mit Determinante}
$\mathbf{A} =
\begin{bmatrix}
    2 & 0 \\
    0 & 2 \\
\end{bmatrix}$ \\
$det(\mathbf{A}) = |\mathbf{A}| = ad - bc = 2\cdot 2 - 0\cdot 0 = 4$ \\
Die Fläche von dieser Matrix ist 4.

\subsection{Determinante 3x3 Matrix}
$\mathbf{A} =
\begin{bmatrix}
    a_1 & b_1 & c_1 \\
    a_2 & b_2 & c_2 \\
    a_3 & b_3 & c_3 \\
\end{bmatrix}$ \\ [7pt]
Volumeninhalt bzw Determinante von $\mathbf{A}$: \\ [7pt]
$a_1b_2c_3 + b_1c_2a_3 + c_1a_2b_3 - c_1b_2a_3 - a_1c_2b_3 - b_1a_2c_3$ \\ [7pt]
Anders kann man auch sagen: Seien $\mathbf{a}$,$\mathbf{b}$ und $\mathbf{c}$ drei Vektoren im Raum. 
Das Volumen von diesen dreie Vektoren aufgespannten Parallelogramm lässt sich so berechnen: \\ [7pt]
$||\mathbf{a}\times \mathbf{b}|| \cdot \mathbf{c}$ \\ [7pt]
Dies entspricht der Determinante der $(3\times 3)$-Matrix deren Spalten $\mathbf{a}$,$\mathbf{b}$ und $\mathbf{c}$ sind (siehe oben).

\subsection{Determinante Regeln}
\begin{itemize}
    \item Determinanten von Matrizen in Dreiecksform: das Produkt der Diagonalelemente. Beispiel:
\end{itemize}
$\mathbf{A} =
\begin{bmatrix}
    2 & 3 & 1 \\
    0 & 1 & 2 \\
    0 & 0 & 5 \\
\end{bmatrix}, det(A)=2\cdot1\cdot5$ \\
\begin{itemize}
    \item Vertauschen zwei Zeilen: ändert das Vorzeichen der Determinante
    \item Multiplikation einer Zeile mit einem Faktor: vergrössert/verkleinert die Determinante um diesen Faktor
    \item Durch die Addition das Vielfache einer Zeile zu einer anderen Zeile ändert sich die Determinante nicht.
    \item Beim Transponieren der Matrizen ändert sich die Determinante nicht. Die oben genannten Regeln gelten auch für die Spalten.
\end{itemize}

\subsection{Determinante äquivalente Aussagen}
Sei $\mathbf{A}$ eine quadratische $(n\times n)$-Matrix:
\begin{itemize}
    \item $\mathbf{A}$ ist regulär
    \item $\mathbf{A^T}$ ist regulär
    \item Die Determinante von $\mathbf{A}$ ist nicht null.
    \item Der Nullvektor ist die einzige Lösung von $\mathbf{Ax}=0$
    \item Der Rang von $\mathbf{A}$ ist gleich $n$
    \item Der algorithmische Test für die Invertierbarkeit ist das Eliminationsverfahren: $\mathbf{A}$ muss $n$
    (von Null verschiedene) Pivotelemente haben
    \item Der algebraische Test für die Invertierbarkeit ist die Determinante von $\mathbf{A}$ : $det(\mathbf{A})$ darf
    nicht Null sein
    \item Die Gleichung, die die Invertierbarkeit tests, ist $\mathbf{Ax}=0$ : $x = 0$ muss die einzige Lösung
    sein.
\end{itemize}




\end{document}