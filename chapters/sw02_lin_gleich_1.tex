\documentclass[../main.tex]{subfiles}


\begin{document}
%%%%%%%%%%%%%%%%%%%%%%%%%%%%%%%%%%%%%%%%%%%%%%%%%%%%%%%%%%%%%%%%%%%%%%%%%%%%%
%                                                                           %
% Lineare Gleichungssysteme lösen 1: Gausselimination, Lösbarkeitskriterien %
%                                                                           %
%%%%%%%%%%%%%%%%%%%%%%%%%%%%%%%%%%%%%%%%%%%%%%%%%%%%%%%%%%%%%%%%%%%%%%%%%%%%%

\chapter{Lineare Gleichungssysteme lösen 1: Gausselimination, Lösbarkeitskriterien}

\section{Begriffe}
\subsection{Leitkoeffizient}
Den ersten Eintrag einer Zeile, der nicht verschwindet, nennt man Leitkoeffizient (hier eingekreist). \\

$\begin{bmatrix}[ccc|c]
    \circled{2} & 5 & -2 & -3 \\
    \circled{3} & 0 & 1 &  7 \\
    0 & 0 & \circled{8} &  10 \\
\end{bmatrix}$

\subsection{Zeilenstufenform ZFS}
Steht in einer Matrix jeder Leitkoeffizient weiter rechts als der Leitkoeffizient in der Zeile darüber
und stehen alle Zeilen ohne Leitkoeffizient (also solche, in denen nur Nullen stehen) ganz unten,
hat die Matrix die Zeilenstufenform (ZFS). (Hier eingekreist)
\begin{itemize}
    \item Die Leitkoeffizienten müssen nicht zwangsläufig alle auf der Hauptdiagonalen stehen.
    \item Eine quadratische Matrix heisst obere Dreieckmatrix, wenn alle Einträge unter der Hauptdiagonalen Null sind. 
    \item Analog dazu definiert man auch eine untere Dreieckmatrix.
\end{itemize}

$\begin{bmatrix}[cccc|c]
    \circled{2} & 5 & -2 & -1 & -3 \\
    0 & 0 & \circled{3} & -2 &  7 \\
    0 & 0 & 0 & \circled{8} &  10 \\
    0 & 0 & 0 & 0 & 0 \\
\end{bmatrix}$

\subsection{Pivotelemente}
Die Leitkoeffiziente, die man durch Vertauschen der Zeilen möglichst weit links bringt,
nenn man Pivotelemente. \\

$\begin{bmatrix}
    0 & 0 & -1 & 5  \\
    2 & -2 & 1 & 3 \\
    4 & -4 & 3 & 8 \\
\end{bmatrix} \rightarrow$
$\begin{bmatrix}
    \circled{2} & -2 & 1 & 3 \\
    0 & 0 & -1 & 5  \\
    4 & -4 & 3 & 8 \\
\end{bmatrix}$

\section{Gauss-Eliminationsverfahren}
Beim Gauss-Verfahren geht es darum, lineare Gleichungssysteme so umzuformen, dass sich eine
Zeilenstufenform ergibt. Dann wird durch Rückwärtseinsetzen eine eindeutige Lösung bestimmt,
falls das LGS lösbar ist. Dabei geht man Schritt für Schritt vor und wendet in jedem Schritt
eine sogenannte elementare Zeilenstufenform (EZ) auf die EKM (erweiterte Koeffizientenmatrix) an. Davon gibt es drei Typen: \\

\begin{enumerate}
    \item Man darf eine Zeile der EKM  mit einem Faktor, der nicht Nul ist, multiplizieren.
    \item Man darf zwei Zeilen der EKM vertauschen.
    \item Man darf das Vielfache einer Zeile zu einer anderen addieren.
\end{enumerate}

\subsection{Beispiel}
Die EKM (erweiterte Koeffizientenmatrix) ist: \\

$\begin{bmatrix}[ccc|c]
    1 & 2 & 1& 2  \\
    3 & 8 & 1 & 12 \\
    0 & 4 & 1 & 2 \\
\end{bmatrix}$ 
\\ [7pt]
Das daraus resultierende LGS (lineare Gleichungssystem): \\
$x + 2y + z = 2$ \\
$3x + 8y + z = 12$ \\
$ 4y + z = 2$
\\ [7pt]
Im EKS alle Elemente unterhalb der Diagonalen Null bekommen (eingekreist): \\ [7pt]
$\begin{bmatrix}[ccc|c]
    1 & 2 & 1& 2  \\
    \circled{3} & 8 & 1 & 12 \\
    \circled{0} & \circled{4} & 1 & 2 \\
\end{bmatrix}$ 
\\ [7pt]
Zweite Zeile minus drei mal die erste Zeile: \\ [7pt]
$\begin{bmatrix}[ccc|c]
    1 & 2 & 1& 2  \\
    0 & 2 & -2 & 6 \\
    0 & 4 & 1 & 2 \\
\end{bmatrix}$ 
\\ [7pt]
Dritte Zeile minus zwei mal die zweite Zeile: \\ [7pt]
$\begin{bmatrix}[ccc|c]
    1 & 2 & 1& 2  \\
    0 & 2 & -2 & 6 \\
    0 & 0 & 5 & -10 \\
\end{bmatrix}$ 
\\ [7pt]
Zweite Zeile durch 2 dividieren, dritte Zeile durch 5 dividieren: \\ [7pt]
$\begin{bmatrix}[ccc|c]
    1 & 2 & 1& 2  \\
    0 & 1 & -1 & 3 \\
    0 & 0 & 1 & -2 \\
\end{bmatrix}$ 
\\ [7pt]
Nun einfaches LGS auflösen: \\ [7pt]
$x+2y+z=2$ \\
$y-z=3$ \\
$z=-2$ \\
$\Rightarrow z=-2; y=1; x=2$

\section{Lösbarkeit eines LGS}
TODO



\end{document}